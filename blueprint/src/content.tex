\chapter{Introduction}

The Chebyshev polynomials of the first kind, denoted $T_N(x)$, form one of the most important families of orthogonal polynomials, with applications spanning approximation theory, numerical analysis, and harmonic analysis.
These polynomials are typically defined either through the trigonometric identity
\begin{equation}
T_N(\cos\theta) = \cos(N\theta),
\end{equation}
or via the three-term recurrence relation
\begin{equation}
T_0(x) = 1, \quad T_1(x) = x, \quad T_{n+1}(x) = 2xT_n(x) - T_{n-1}(x).
\end{equation}

In this project, we present a novel geometric construction of Chebyshev polynomials using rotated roots of unity. Consider the $N$-th roots of unity on the complex unit circle:
\begin{equation}
\omega_k = \e^{2\pi i k/N}, \quad k = 0, 1, \ldots, N-1.
\end{equation}
When these roots are rotated by an angle $\varphi$ (equivalently, multiplied by $\e^{i\varphi}$) and projected onto the real axis, we obtain the $N$ real numbers
\begin{equation}\label{eq:rotated_roots}
r_k(\varphi) = \cos\left(\varphi + \frac{2\pi k}{N}\right), \quad k = 0, 1, \ldots, N-1.
\end{equation}

Let $P_N(x; \varphi)$ denote the monic polynomial having these projected values as roots:
\begin{equation}\label{eq:monic_poly}
P_N(x; \varphi) = \prod_{k=0}^{N-1} \left(x - \cos\left(\varphi + \frac{2\pi k}{N}\right)\right).
\end{equation}

To match the normalization of Chebyshev polynomials, which have leading coefficient $2^{N-1}$ for $N \geq 1$, we define the scaled polynomial:
\begin{equation}\label{eq:scaled_poly}
S_N(x; \varphi) = 2^{N-1} \cdot P_N(x; \varphi).
\end{equation}

Our main result establishes that this construction yields the Chebyshev polynomial up to an additive constant.

\begin{theorem}[Main Theorem]\label{thm:main}
\lean{rotated_roots_yield_chebyshev}
\leanok
For any positive integer $N \geq 1$ and any angle $\varphi \in \R$, there exists a constant $c(\varphi) \in \R$ such that
\begin{equation}\label{eq:main_result}
S_N(x; \varphi) = T_N(x) + c(\varphi).
\end{equation}
Moreover, all coefficients of $S_N(x; \varphi)$ of degree $k \geq 1$ are independent of $\varphi$ and equal the corresponding coefficients of $T_N(x)$.
\uses{lem:power_sum_invariance, lem:newton_identities, lem:chebyshev_roots}
\end{theorem}

\section{Significance and context}

This theorem reveals several deep connections:

\begin{enumerate}
\item \textbf{Discrete geometry meets polynomial theory.} The result shows that Chebyshev polynomials emerge naturally from the simplest discrete geometric object---equally-spaced points on a circle---through elementary operations (rotation and projection).

\item \textbf{Power sum invariance.} The proof hinges on showing that certain discrete sums of powers of cosines are invariant under the phase shift $\varphi$. This invariance reflects discrete orthogonality properties of trigonometric functions, a finite analog of continuous Fourier analysis.

\item \textbf{Symmetric function theory.} Newton's identities provide the bridge from power sums to elementary symmetric polynomials, and hence to polynomial coefficients. The theorem demonstrates this classical algebraic machinery in action.

\item \textbf{Roots and quadrature.} The projected roots $r_k(0) = \cos(2\pi k/N)$ are closely related to the Chebyshev-Gauss quadrature nodes $\cos((2k+1)\pi/(2N))$, both arising from equidistributed points on the circle.
\end{enumerate}

\section{Proof strategy}

The proof of Theorem~\ref{thm:main} proceeds through several key steps:

\begin{enumerate}
\item \textbf{Discrete orthogonality.} We establish that sums of complex exponentials and cosines over the $N$-th roots of unity satisfy discrete orthogonality relations analogous to Fourier orthogonality.

\item \textbf{Power sum invariance.} Using binomial expansion of $\cos^j(\varphi + 2\pi k/N)$ and the orthogonality relations, we prove that the power sums
\begin{equation}
\sum_{k=0}^{N-1} \cos^j\left(\varphi + \frac{2\pi k}{N}\right)
\end{equation}
are independent of $\varphi$ for all $1 \leq j < N$.

\item \textbf{Newton's identities.} We apply Newton's identities to show that if two sets of roots have identical power sums (except for the 0-th power sum, which counts the number of roots), their elementary symmetric polynomials match. Consequently, the monic polynomials with these roots have identical coefficients except possibly the constant term.

\item \textbf{Chebyshev characterization.} We identify the roots of $T_N(x)$ as $\cos((2k + 1)\pi/(2N))$ for $k = 0, \ldots, N-1$, and verify that these roots have the same power sums as the rotated roots $r_k(\varphi)$. This establishes coefficient-wise equality for degrees $k \geq 1$.

\item \textbf{Special case $\varphi = 0$.} For $\varphi = 0$, explicit calculation shows $S_N(x; 0) = T_N(x)$, confirming the identification up to the constant term.
\end{enumerate}

\chapter{Preliminaries}

\section{Chebyshev polynomials}

The Chebyshev polynomials of the first kind $T_N(x)$ are defined by the relation
\begin{equation}
T_N(\cos \theta) = \cos(N\theta).
\end{equation}
This uniquely determines $T_N$ as a polynomial of degree $N$. The first few Chebyshev polynomials are:
\begin{align*}
T_0(x) &= 1, \\
T_1(x) &= x, \\
T_2(x) &= 2x^2 - 1, \\
T_3(x) &= 4x^3 - 3x, \\
T_4(x) &= 8x^4 - 8x^2 + 1.
\end{align*}

\begin{proposition}\label{prop:chebyshev_properties}
\lean{Polynomial.Chebyshev.T}
\leanok
The Chebyshev polynomials satisfy the following properties:
\begin{enumerate}[(a)]
\item Recurrence: $T_{n+1}(x) = 2xT_n(x) - T_{n-1}(x)$ for $n \geq 1$.
\item Leading coefficient: The leading coefficient of $T_N(x)$ is $2^{N-1}$ for $N \geq 1$.
\item Roots: The roots of $T_N(x)$ are
\begin{equation}\label{eq:chebyshev_roots}
x_k = \cos\left(\frac{(2k + 1)\pi}{2N}\right), \quad k = 0, 1, \ldots, N-1.
\end{equation}
\item Extrema: On $[-1, 1]$, $T_N$ attains its extremal values $\pm 1$ at the $N + 1$ points $\cos(j\pi/N)$ for $j = 0, 1, \ldots, N$.
\end{enumerate}
\end{proposition}

\section{Roots of unity and discrete Fourier analysis}

The $N$-th roots of unity are the solutions to $z^N = 1$ in $\C$:
\begin{equation}
\omega_k = \e^{2\pi i k/N}, \quad k = 0, 1, \ldots, N-1.
\end{equation}
A primitive $N$-th root of unity is one that generates the cyclic group of all $N$-th roots under multiplication; $\omega = \e^{2\pi i/N}$ is primitive, and $\omega_k = \omega^k$.

The fundamental orthogonality relation is:

\begin{lemma}[Discrete orthogonality]\label{lem:discrete_orthogonality}
\lean{geom_sum_eq}
\leanok
For any integer $m$,
\begin{equation}
\sum_{k=0}^{N-1} \e^{2\pi i m k/N} = \begin{cases} N & \text{if } N \mid m, \\ 0 & \text{otherwise.} \end{cases}
\end{equation}
\end{lemma}

\section{Power sums and elementary symmetric polynomials}

Let $\alpha_1, \ldots, \alpha_N$ be a collection of real or complex numbers (possibly with repetitions). The \emph{power sums} are defined by
\begin{equation}
p_j = \sum_{i=1}^N \alpha_i^j, \quad j \geq 0.
\end{equation}

The \emph{elementary symmetric polynomials} are the coefficients appearing in the factorization
\begin{equation}
\prod_{i=1}^N (t - \alpha_i) = t^N - e_1 t^{N-1} + e_2 t^{N-2} - \cdots + (-1)^N e_N,
\end{equation}
where
\begin{equation}
e_k = \sum_{1 \leq i_1 < \cdots < i_k \leq N} \alpha_{i_1} \cdots \alpha_{i_k}.
\end{equation}

\begin{theorem}[Newton's identities]\label{thm:newton_identities}
\lean{newton_power_sum_eq_esymm}
\leanok
The power sums and elementary symmetric polynomials are related by:
\begin{equation}
k \cdot e_k = \sum_{i=1}^k (-1)^{i-1} e_{k-i} p_i, \quad k = 1, 2, \ldots, N,
\end{equation}
where we set $e_0 = 1$.
\uses{lem:discrete_orthogonality}
\end{theorem}

An immediate consequence is that the elementary symmetric polynomials (and hence the coefficients of the monic polynomial with roots $\alpha_1, \ldots, \alpha_N$) are determined by the power sums.

\begin{corollary}\label{cor:power_sums_determine_coefficients}
\leanok
Let $\alpha_1, \ldots, \alpha_N$ and $\beta_1, \ldots, \beta_N$ be two collections of numbers. If
\begin{equation}
\sum_{i=1}^N \alpha_i^j = \sum_{i=1}^N \beta_i^j \quad \text{for all } j = 1, 2, \ldots, N,
\end{equation}
then the monic polynomials with roots $\{\alpha_i\}$ and $\{\beta_i\}$ have identical coefficients of degree $k \geq 1$. They may differ in the constant term if $\sum \alpha_i \neq \sum \beta_i$ (i.e., if $p_0$ differs).
\uses{thm:newton_identities}
\end{corollary}

\chapter{Discrete Orthogonality Relations}

The discrete orthogonality of roots of unity (Lemma~\ref{lem:discrete_orthogonality}) has powerful consequences for sums of trigonometric functions evaluated at equally-spaced angles. We develop these consequences here.

\section{Sums of cosines at rotated roots}

\begin{lemma}\label{lem:sum_cos_phi_vanishes}
\lean{sum_cos_rotated_roots}
\leanok
For any $N \geq 2$ and any $\varphi \in \R$,
\begin{equation}
\sum_{k=0}^{N-1} \cos\left(\varphi + \frac{2\pi k}{N}\right) = 0.
\end{equation}
\uses{lem:discrete_orthogonality}
\end{lemma}

\begin{lemma}\label{lem:sum_cos_m_phi_vanishes}
\lean{sum_cos_m_rotated_roots}
\leanok
For any integers $N \geq 1$, $m$ with $0 < m < N$, and any $\varphi \in \R$,
\begin{equation}
\sum_{k=0}^{N-1} \cos\left(m\left(\varphi + \frac{2\pi k}{N}\right)\right) = 0.
\end{equation}
\uses{lem:discrete_orthogonality}
\end{lemma}

\section{Chebyshev angle sums}

The Chebyshev roots are located at angles $\theta_k = (2k + 1)\pi/(2N)$. Sums over these angles require more care due to the non-uniform phase.

\begin{lemma}\label{lem:chebyshev_angle_sum_odd}
\leanok
For odd $m$ with $0 < m < N$,
\begin{equation}
\sum_{k=0}^{N-1} \cos\left(m \cdot \frac{(2k + 1)\pi}{2N}\right) = 0.
\end{equation}
\uses{lem:discrete_orthogonality}
\end{lemma}

\begin{lemma}\label{lem:chebyshev_angle_sum_even}
\leanok
For even $m$ with $0 < m < N$,
\begin{equation}
\sum_{k=0}^{N-1} \cos\left(m \cdot \frac{(2k + 1)\pi}{2N}\right) = 0.
\end{equation}
\uses{lem:discrete_orthogonality}
\end{lemma}

\chapter{Power Sum Invariance}

We now prove that the power sums of the rotated projections $r_k(\varphi) = \cos(\varphi + 2\pi k/N)$ are independent of $\varphi$.

\begin{theorem}[Power sum invariance]\label{lem:power_sum_invariance}
\lean{powerSum_rotated_roots_indep_phi}
\leanok
For any integers $N \geq 1$, $j$ with $1 \leq j < N$, and any $\varphi \in \R$,
\begin{equation}
\sum_{k=0}^{N-1} \cos^j\left(\varphi + \frac{2\pi k}{N}\right)
\end{equation}
is independent of $\varphi$.
\uses{lem:sum_cos_m_phi_vanishes}
\end{theorem}

\chapter{From Power Sums to Polynomial Coefficients}

Having established that the power sums are $\varphi$-independent, we now apply Newton's identities to deduce that the polynomial coefficients (except the constant term) are also $\varphi$-independent.

\begin{theorem}\label{thm:coefficients_independent}
\leanok
For any $N \geq 1$ and any $k$ with $1 \leq k \leq N$, the coefficient of $x^k$ in $S_N(x; \varphi)$ is independent of $\varphi$.
\uses{lem:power_sum_invariance, thm:newton_identities}
\end{theorem}

\chapter{Proof of Main Theorem}

We now assemble the pieces to prove Theorem~\ref{thm:main}.

\section{Chebyshev roots and their power sums}

\begin{lemma}\label{lem:chebyshev_roots}
\lean{powerSum_chebyshev_roots}
\leanok
Let $\xi_k = \cos((2k + 1)\pi/(2N))$ for $k = 0, \ldots, N-1$ denote the roots of $T_N(x)$. Then for any $1 \leq j < N$,
\begin{equation}
\sum_{k=0}^{N-1} \xi_k^j = \begin{cases} \frac{N}{2^j} \binom{j}{j/2} & \text{if } j \text{ is even}, \\ 0 & \text{if } j \text{ is odd.} \end{cases}
\end{equation}
\uses{lem:chebyshev_angle_sum_odd, lem:chebyshev_angle_sum_even}
\end{lemma}

\begin{theorem}[Power sum equality]\label{thm:power_sum_equality}
\leanok
For $1 \leq j < N$,
\begin{equation}
\sum_{k=0}^{N-1} \cos^j\left(\varphi + \frac{2\pi k}{N}\right) = \sum_{k=0}^{N-1} \cos^j\left(\frac{(2k + 1)\pi}{2N}\right).
\end{equation}
\uses{lem:power_sum_invariance, lem:chebyshev_roots}
\end{theorem}

\section{Completion of the proof}

\begin{proof}[Proof of Theorem~\ref{thm:main}]
Let $r_k(\varphi) = \cos(\varphi + 2\pi k/N)$ for $k = 0, \ldots, N-1$ be the roots of $P_N(x; \varphi)$, and let $\xi_k = \cos((2k + 1)\pi/(2N))$ be the roots of $T_N(x)$.

By Theorem~\ref{thm:power_sum_equality}, the power sums of $\{r_k(\varphi)\}$ and $\{\xi_k\}$ are equal for all $j$ with $1 \leq j < N$.

By Newton's identities (Theorem~\ref{thm:newton_identities}), the elementary symmetric polynomials $e_k$ for $k = 1, \ldots, N$ are determined by the power sums $p_1, \ldots, p_N$. Since the power sums match, the elementary symmetric polynomials match:
\begin{equation}
e_k(r_0(\varphi), \ldots, r_{N-1}(\varphi)) = e_k(\xi_0, \ldots, \xi_{N-1}), \quad k = 1, \ldots, N.
\end{equation}

The monic polynomial $P_N(x; \varphi)$ has roots $r_k(\varphi)$ and the Chebyshev polynomial $T_N(x)$ (when divided by its leading coefficient $2^{N-1}$) has roots $\xi_k$. By Vieta's formulas, the coefficient of $x^{N-k}$ in $P_N(x; \varphi)$ is $(-1)^k e_k(r_0, \ldots, r_{N-1})$, and similarly for the monic version of $T_N(x)$.

Thus, $P_N(x; \varphi)$ and $T_N(x)/2^{N-1}$ have the same coefficients for all degrees $k = 1, \ldots, N$. They may differ in the constant term. Multiplying both by $2^{N-1}$:
\begin{equation}
S_N(x; \varphi) = 2^{N-1} P_N(x; \varphi) \quad \text{and} \quad T_N(x)
\end{equation}
have the same coefficients for degrees $k = 1, \ldots, N$. They may differ in the constant term.

Hence there exists $c(\varphi) \in \R$ such that
\begin{equation}
S_N(x; \varphi) = T_N(x) + c(\varphi),
\end{equation}
completing the proof.
\end{proof}

\chapter{Introduction}

\textbf{PDF Version:} A compiled PDF version of this blueprint is available at \texttt{print/blueprint.pdf}.

The Chebyshev polynomials of the first kind, denoted $T_N(x)$, form one of the most important families of orthogonal polynomials, with applications spanning approximation theory, numerical analysis, and harmonic analysis.
These polynomials are typically defined either through the trigonometric identity
\begin{equation}
T_N(\cos\theta) = \cos(N\theta),
\end{equation}
or via the three-term recurrence relation
\begin{equation}
T_0(x) = 1, \quad T_1(x) = x, \quad T_{n+1}(x) = 2xT_n(x) - T_{n-1}(x).
\end{equation}

In this project, we present a novel geometric construction of Chebyshev polynomials using rotated roots of unity. Consider the $N$-th roots of unity on the complex unit circle:
\begin{equation}
\omega_k = \e^{2\pi i k/N}, \quad k = 0, 1, \ldots, N-1.
\end{equation}
These form a regular $N$-gon inscribed in the unit circle, with vertices equally spaced at angular intervals of $2\pi/N$.

When these roots are rotated by an angle $\varphi$ (equivalently, multiplied by $\e^{i\varphi}$) and projected onto the real axis, we obtain the $N$ real numbers
\begin{equation}\label{eq:rotated_roots}
r_k(\varphi) = \cos\left(\varphi + \frac{2\pi k}{N}\right), \quad k = 0, 1, \ldots, N-1.
\end{equation}
\lean{rotatedRootsOfUnity, realProjections, realProjectionsList}
\leanok

The rotation angle $\varphi$ controls the orientation of the $N$-gon, while the projection onto the real axis yields a multiset of real values in $[-1, 1]$.
\lean{realProjection_eq_cos, realProjection_mem_Icc}
\leanok
As $\varphi$ varies from $0$ to $2\pi$, the projected roots sweep through all possible configurations of $N$ points on $[-1, 1]$ constrained by the symmetry of the roots of unity.
\lean{rotatedRoot_eq_exp, card_realProjectionsList}
\leanok

Let $P_N(x; \varphi)$ denote the monic polynomial having these projected values as roots:
\begin{equation}\label{eq:monic_poly}
P_N(x; \varphi) = \prod_{k=0}^{N-1} \left(x - \cos\left(\varphi + \frac{2\pi k}{N}\right)\right).
\end{equation}
\lean{polynomialFromRealRoots, unscaledPolynomial}
\leanok
\uses{rotatedRootsOfUnity, realProjectionsList}

Since the roots are real and lie in $[-1, 1]$, this polynomial has real coefficients and degree $N$.
\lean{polynomialFromRealRoots_degree, unscaledPolynomial_degree, unscaledPolynomial_monic}
\leanok
Moreover, for each root $r_k(\varphi)$, we have $P_N(r_k(\varphi); \varphi) = 0$.
\lean{polynomialFromRealRoots_eval_mem, realProjection_mem_list, unscaledPolynomial_eval_at_projection}
\leanok

To match the normalization of Chebyshev polynomials, which have leading coefficient $2^{N-1}$ for $N \geq 1$, we define the scaled polynomial:
\begin{equation}\label{eq:scaled_poly}
S_N(x; \varphi) = 2^{N-1} \cdot P_N(x; \varphi).
\end{equation}
\lean{scaledPolynomial}
\leanok
\uses{unscaledPolynomial}

The scaled polynomial inherits the degree and roots from $P_N$:
\lean{scaledPolynomial_degree, scaledPolynomial_leadingCoeff, scaledPolynomial_eval_at_projection}
\leanok

Our main result establishes that this construction yields the Chebyshev polynomial up to an additive constant.

\begin{theorem}[Main Theorem]\label{thm:main}
\lean{rotated_roots_yield_chebyshev, rotated_roots_coeffs_match_chebyshev}
\leanok
For any positive integer $N \geq 1$ and any angle $\varphi \in \R$, there exists a constant $c(\varphi) \in \R$ such that
\begin{equation}\label{eq:main_result}
S_N(x; \varphi) = T_N(x) + c(\varphi).
\end{equation}
Moreover, all coefficients of $S_N(x; \varphi)$ of degree $k \geq 1$ are independent of $\varphi$ and equal the corresponding coefficients of $T_N(x)$.
\lean{constant_term_only_varies}
\leanok
\uses{lem:power_sum_invariance, thm:newton_identities, lem:chebyshev_roots, scaledPolynomial}

\begin{remark}
\textit{(See \href{/docs/ChebyshevCircles/MainTheorem.html\#rotated_roots_yield_chebyshev}{API Documentation} | \href{../paper/chebyshev_circles.pdf}{Paper Theorem 1.1})}
\end{remark}
\end{theorem}

\section{Significance and context}

This theorem reveals several deep connections:

\begin{enumerate}
\item \textbf{Discrete geometry meets polynomial theory.} The result shows that Chebyshev polynomials emerge naturally from the simplest discrete geometric object---equally-spaced points on a circle---through elementary operations (rotation and projection).

\item \textbf{Power sum invariance.} The proof hinges on showing that certain discrete sums of powers of cosines are invariant under the phase shift $\varphi$. This invariance reflects discrete orthogonality properties of trigonometric functions, a finite analog of continuous Fourier analysis.

\item \textbf{Symmetric function theory.} Newton's identities provide the bridge from power sums to elementary symmetric polynomials, and hence to polynomial coefficients. The theorem demonstrates this classical algebraic machinery in action.

\item \textbf{Roots and quadrature.} The projected roots $r_k(0) = \cos(2\pi k/N)$ are closely related to the Chebyshev-Gauss quadrature nodes $\cos((2k+1)\pi/(2N))$, both arising from equidistributed points on the circle.
\end{enumerate}

\section{Computational illustration: N = 5}

To make the construction concrete, consider $N = 5$ with $\varphi = 0$. The five roots of unity are
\begin{equation}
\omega_k = \e^{2\pi i k/5}, \quad k = 0, 1, 2, 3, 4,
\end{equation}
positioned at angles $0°, 72°, 144°, 216°, 288°$ on the unit circle. Their real projections are
\begin{equation}
r_k(0) = \cos(2\pi k/5) = \{1, 0.309, -0.809, -0.809, 0.309\}.
\end{equation}
The monic polynomial with these roots is
\begin{equation}
P_5(x; 0) = (x - 1)(x - 0.309)(x + 0.809)^2(x - 0.309),
\end{equation}
which simplifies to a polynomial of degree 5. Scaling by $2^4 = 16$ gives
\begin{equation}
S_5(x; 0) = 16x^5 - 20x^3 + 5x + c,
\end{equation}
where the coefficients of $x^5, x^3, x$ match those of $T_5(x) = 16x^5 - 20x^3 + 5x$, differing only in the constant term.

Rotating by $\varphi = \pi/10$ shifts the roots but preserves the polynomial coefficients (except the constant), demonstrating the invariance principle.

\section{Proof strategy}

The proof of Theorem~\ref{thm:main} proceeds through several key steps:

\begin{enumerate}
\item \textbf{Discrete orthogonality.} We establish that sums of complex exponentials and cosines over the $N$-th roots of unity satisfy discrete orthogonality relations analogous to Fourier orthogonality.

\item \textbf{Power sum invariance.} Using binomial expansion of $\cos^j(\varphi + 2\pi k/N)$ and the orthogonality relations, we prove that the power sums
\begin{equation}
\sum_{k=0}^{N-1} \cos^j\left(\varphi + \frac{2\pi k}{N}\right)
\end{equation}
are independent of $\varphi$ for all $1 \leq j < N$.

\item \textbf{Newton's identities.} We apply Newton's identities to show that if two sets of roots have identical power sums (except for the 0-th power sum, which counts the number of roots), their elementary symmetric polynomials match. Consequently, the monic polynomials with these roots have identical coefficients except possibly the constant term.

\item \textbf{Chebyshev characterization.} We identify the roots of $T_N(x)$ as $\cos((2k + 1)\pi/(2N))$ for $k = 0, \ldots, N-1$, and verify that these roots have the same power sums as the rotated roots $r_k(\varphi)$. This establishes coefficient-wise equality for degrees $k \geq 1$.

\item \textbf{Special case $\varphi = 0$.} For $\varphi = 0$, explicit calculation shows $S_N(x; 0) = T_N(x)$, confirming the identification up to the constant term.
\end{enumerate}

\chapter{Preliminaries}

\section{Chebyshev polynomials}

The Chebyshev polynomials of the first kind $T_N(x)$ are defined by the relation
\begin{equation}
T_N(\cos \theta) = \cos(N\theta).
\end{equation}
This uniquely determines $T_N$ as a polynomial of degree $N$. The first few Chebyshev polynomials are:
\begin{align*}
T_0(x) &= 1, \\
T_1(x) &= x, \\
T_2(x) &= 2x^2 - 1, \\
T_3(x) &= 4x^3 - 3x, \\
T_4(x) &= 8x^4 - 8x^2 + 1.
\end{align*}

\begin{proposition}\label{prop:chebyshev_properties}
\lean{Polynomial.Chebyshev.T}
\leanok
The Chebyshev polynomials satisfy the following properties:
\begin{enumerate}[(a)]
\item Recurrence: $T_{n+1}(x) = 2xT_n(x) - T_{n-1}(x)$ for $n \geq 1$.
\lean{Polynomial.Chebyshev.T_add_two}
\leanok
\item Leading coefficient: The leading coefficient of $T_N(x)$ is $2^{N-1}$ for $N \geq 1$.
\lean{chebyshev_T_leadingCoeff}
\leanok
\item Roots: The roots of $T_N(x)$ are
\begin{equation}\label{eq:chebyshev_roots}
x_k = \cos\left(\frac{(2k + 1)\pi}{2N}\right), \quad k = 0, 1, \ldots, N-1.
\end{equation}
\lean{chebyshevRoot, chebyshevRootsList, chebyshev_T_eval_chebyshevRoot, chebyshev_T_eval_eq_zero_iff}
\leanok
\item Extrema: On $[-1, 1]$, $T_N$ attains its extremal values $\pm 1$ at the $N + 1$ points $\cos(j\pi/N)$ for $j = 0, 1, \ldots, N$.
\end{enumerate}
\end{proposition}

\section{Roots of unity and discrete Fourier analysis}

The $N$-th roots of unity are the solutions to $z^N = 1$ in $\C$:
\begin{equation}
\omega_k = \e^{2\pi i k/N}, \quad k = 0, 1, \ldots, N-1.
\end{equation}
These form a cyclic group under multiplication, isomorphic to $\mathbb{Z}/N\mathbb{Z}$. A primitive $N$-th root of unity is one that generates the cyclic group of all $N$-th roots under multiplication; $\omega = \e^{2\pi i/N}$ is primitive, and $\omega_k = \omega^k$.

Geometrically, the $N$-th roots of unity are equally-spaced points on the unit circle in the complex plane, with angular separation $2\pi/N$. Their symmetry under rotation by $2\pi/N$ is the source of all discrete orthogonality relations we exploit.

The fundamental orthogonality relation is:

\begin{lemma}[Discrete orthogonality]\label{lem:discrete_orthogonality}
\lean{IsPrimitiveRoot.geom_sum_eq_zero, mul_geom_sum}
\leanok
For any integer $m$,
\begin{equation}
\sum_{k=0}^{N-1} \e^{2\pi i m k/N} = \begin{cases} N & \text{if } N \mid m, \\ 0 & \text{otherwise.} \end{cases}
\end{equation}
\lean{sum_primitive_root_pow_mul, sum_pow_primitive_root_mul}
\leanok
\uses{primitiveRoots}

\textit{(See \href{/docs/ChebyshevCircles/ChebyshevOrthogonality.html}{API Documentation} | \href{../paper/chebyshev_circles.pdf}{Paper Lemma 2.2})}
\end{lemma}

\begin{proof}
If $N \mid m$, then $\e^{2\pi i m k/N} = 1$ for all $k$, so the sum equals $N$.

If $N \nmid m$, let $\omega = \e^{2\pi i m/N} \neq 1$. Then the sum is a geometric series:
\begin{equation}
\sum_{k=0}^{N-1} \omega^k = \frac{\omega^N - 1}{\omega - 1} = \frac{1 - 1}{\omega - 1} = 0,
\end{equation}
since $\omega^N = \e^{2\pi i m} = 1$.
\end{proof}

This lemma is the discrete analog of the orthogonality relation for continuous exponentials:
\begin{equation}
\frac{1}{2\pi}\int_0^{2\pi} \e^{i m \theta} \, d\theta = \begin{cases} 1 & \text{if } m = 0, \\ 0 & \text{otherwise.} \end{cases}
\end{equation}
The discrete sum with $N$ terms approximates the integral as $N \to \infty$, providing a bridge between discrete and continuous Fourier analysis.

\section{Power sums and elementary symmetric polynomials}

Let $\alpha_1, \ldots, \alpha_N$ be a collection of real or complex numbers (possibly with repetitions). The \emph{power sums} are defined by
\begin{equation}
p_j = \sum_{i=1}^N \alpha_i^j, \quad j \geq 0.
\end{equation}

The \emph{elementary symmetric polynomials} are the coefficients appearing in the factorization
\begin{equation}
\prod_{i=1}^N (t - \alpha_i) = t^N - e_1 t^{N-1} + e_2 t^{N-2} - \cdots + (-1)^N e_N,
\end{equation}
where
\begin{equation}
e_k = \sum_{1 \leq i_1 < \cdots < i_k \leq N} \alpha_{i_1} \cdots \alpha_{i_k}.
\end{equation}

\begin{theorem}[Newton's identities]\label{thm:newton_identities}
\lean{MvPolynomial.mul_esymm_eq_sum, multiset_newton_identity}
\leanok
The power sums and elementary symmetric polynomials are related by:
\begin{equation}
k \cdot e_k = \sum_{i=1}^k (-1)^{i-1} e_{k-i} p_i, \quad k = 1, 2, \ldots, N,
\end{equation}
where we set $e_0 = 1$.
\lean{esymm_eq_of_psum_eq, esymm_rotated_roots_invariant}
\leanok
\uses{lem:discrete_orthogonality}

\textit{(See \href{/docs/ChebyshevCircles/NewtonIdentities.html}{API Documentation} | \href{../paper/chebyshev_circles.pdf}{Paper Theorem 2.3})}
\end{theorem}

An immediate consequence is that the elementary symmetric polynomials (and hence the coefficients of the monic polynomial with roots $\alpha_1, \ldots, \alpha_N$) are determined by the power sums.

\begin{corollary}\label{cor:power_sums_determine_coefficients}
\lean{polynomial_coeff_eq_of_esymm_eq}
\leanok
Let $\alpha_1, \ldots, \alpha_N$ and $\beta_1, \ldots, \beta_N$ be two collections of numbers. If
\begin{equation}
\sum_{i=1}^N \alpha_i^j = \sum_{i=1}^N \beta_i^j \quad \text{for all } j = 1, 2, \ldots, N,
\end{equation}
then the monic polynomials with roots $\{\alpha_i\}$ and $\{\beta_i\}$ have identical coefficients of degree $k \geq 1$. They may differ in the constant term if $\sum \alpha_i \neq \sum \beta_i$ (i.e., if $p_0$ differs).
\lean{Multiset.prod_X_sub_C_coeff}
\leanok
\uses{thm:newton_identities, esymm_eq_of_psum_eq}
\end{corollary}

\begin{proof}
By Theorem~\ref{thm:newton_identities}, if $p_j(\alpha) = p_j(\beta)$ for $j = 1, \ldots, N$, then $e_k(\alpha) = e_k(\beta)$ for $k = 1, \ldots, N$. By Vieta's formulas, the coefficient of $x^{N-k}$ in the monic polynomial $\prod(x - \alpha_i)$ is $(-1)^k e_k$. Thus the coefficients match for $k \geq 1$. The constant term $(-1)^N e_N$ involves the product of all roots, which may differ.
\lean{esymm_eq_of_psum_eq, multiset_esymm_one_eq_sum}
\leanok
\end{proof}

\chapter{Discrete Orthogonality Relations}

The discrete orthogonality of roots of unity (Lemma~\ref{lem:discrete_orthogonality}) has powerful consequences for sums of trigonometric functions evaluated at equally-spaced angles. We develop these consequences here.

\section{Sums of cosines at rotated roots}

The discrete orthogonality of roots of unity manifests geometrically: the projections of equally-spaced points on the circle always balance around the origin.

\begin{lemma}\label{lem:sum_cos_phi_vanishes}
\lean{sum_cos_roots_of_unity}
\leanok
For any $N \geq 2$ and any $\varphi \in \R$,
\begin{equation}
\sum_{k=0}^{N-1} \cos\left(\varphi + \frac{2\pi k}{N}\right) = 0.
\end{equation}
\lean{realProjectionsList_sum}
\leanok
\uses{lem:discrete_orthogonality, Complex.exp_ofReal_mul_I_re, Complex.isPrimitiveRoot_exp}

\textit{(See Lemma 3.1 in \href{../paper/chebyshev_circles.pdf}{the paper})}
\end{lemma}

\begin{proof}
We use Euler's formula: $\cos\theta = \Re(\e^{i\theta})$. Thus
\begin{equation}
\sum_{k=0}^{N-1} \cos\left(\varphi + \frac{2\pi k}{N}\right) = \Re\left(\sum_{k=0}^{N-1} \e^{i(\varphi + 2\pi k/N)}\right) = \Re\left(\e^{i\varphi} \sum_{k=0}^{N-1} \e^{2\pi i k/N}\right).
\end{equation}
By Lemma~\ref{lem:discrete_orthogonality} with $m = 1$ (noting that $N \nmid 1$ for $N \geq 2$), the inner sum vanishes. Hence the entire expression is zero.
\end{proof}

\begin{lemma}\label{lem:sum_cos_m_phi_vanishes}
\lean{sum_cos_multiple_rotated_roots, sum_cos_int_multiple_vanishes}
\leanok
For any integers $N \geq 1$, $m$ with $0 < m < N$, and any $\varphi \in \R$,
\begin{equation}
\sum_{k=0}^{N-1} \cos\left(m\left(\varphi + \frac{2\pi k}{N}\right)\right) = 0.
\end{equation}
\uses{lem:discrete_orthogonality, sum_primitive_root_pow_mul}

\textit{(See Lemma 3.2 in \href{../paper/chebyshev_circles.pdf}{the paper})}
\end{lemma}

\begin{proof}
Using Euler's formula, $\cos(m\theta) = \Re(\e^{im\theta})$, we have
\begin{equation}
\sum_{k=0}^{N-1} \cos\left(m\left(\varphi + \frac{2\pi k}{N}\right)\right) = \Re\left(\e^{im\varphi} \sum_{k=0}^{N-1} \e^{2\pi i mk/N}\right).
\end{equation}
Since $0 < m < N$, we have $N \nmid m$, so the inner sum vanishes by Lemma~\ref{lem:discrete_orthogonality}.
\lean{cos_as_exp}
\leanok
\end{proof}

\subsection{Connection to discrete Fourier transform}

The vanishing sums in Lemmas~\ref{lem:sum_cos_phi_vanishes} and~\ref{lem:sum_cos_m_phi_vanishes} reflect the orthogonality of the discrete Fourier basis. Define the discrete Fourier transform of a sequence $(a_0, \ldots, a_{N-1})$ as
\begin{equation}
\hat{a}_m = \sum_{k=0}^{N-1} a_k \e^{-2\pi i mk/N}.
\end{equation}
Our lemmas show that the constant sequence $a_k = \e^{i\varphi}$ (rotated unit phasor) has Fourier coefficients $\hat{a}_m = 0$ for $0 < m < N$ and $\hat{a}_0 = N \e^{i\varphi}$. This concentration of energy in the DC component (zero frequency) is characteristic of constant signals.

Similarly, for the sequence $b_k = \e^{im(\varphi + 2\pi k/N)}$ (complex exponential at frequency $m$), the projection onto the real axis (cosine) integrates to zero when summed uniformly over the roots of unity. This is the discrete analog of
\begin{equation}
\int_0^{2\pi} \cos(m\theta) \, d\theta = 0 \quad \text{for } m \neq 0.
\end{equation}

\section{Chebyshev angle sums}

The Chebyshev roots are located at angles $\theta_k = (2k + 1)\pi/(2N)$. Unlike the equidistributed roots of unity, these angles carry a phase offset of $\pi/(2N)$. Nevertheless, discrete orthogonality still forces their trigonometric sums to vanish.

\begin{lemma}\label{lem:chebyshev_angle_sum_odd}
\lean{ChebyshevCircles.sum_cos_chebyshev_angles_vanishes}
\leanok
For odd $m$ with $0 < m < N$,
\begin{equation}
\sum_{k=0}^{N-1} \cos\left(m \cdot \frac{(2k + 1)\pi}{2N}\right) = 0.
\end{equation}
\uses{lem:discrete_orthogonality, Real.cos_int_mul_pi_sub}
\end{lemma}

\begin{proof}[Sketch]
For odd $m$, we exploit the symmetry $\cos(m\pi - \alpha) = -\cos(\alpha)$ when $m$ is odd. Pairing terms $k$ and $N - k - 1$ yields cancellation. The formalization uses trigonometric reflection identities.
\lean{Real.cos_int_mul_pi_sub}
\leanok
\end{proof}

\begin{lemma}\label{lem:chebyshev_angle_sum_even}
\lean{ChebyshevCircles.sum_cos_chebyshev_angles_even_vanishes}
\leanok
For even $m$ with $0 < m < N$,
\begin{equation}
\sum_{k=0}^{N-1} \cos\left(m \cdot \frac{(2k + 1)\pi}{2N}\right) = 0.
\end{equation}
\lean{ChebyshevCircles.binomial_terms_vanish_chebyshev}
\leanok
\uses{lem:discrete_orthogonality, sum_exp_chebyshev_angles}
\end{lemma}

\begin{proof}[Sketch]
For even $m = 2\ell$, convert to complex exponentials. The sum becomes
\begin{equation}
\Re\left(\e^{i\ell\pi/N} \sum_{k=0}^{N-1} \e^{2\pi i \ell k/N}\right).
\end{equation}
Since $0 < \ell < N$, the geometric sum vanishes by Lemma~\ref{lem:discrete_orthogonality}.
\lean{sum_exp_chebyshev_angles}
\leanok
\end{proof}

\subsection{Geometric interpretation}

The vanishing of these angle sums has a simple geometric interpretation. The angles $(2k+1)\pi/(2N)$ for $k = 0, \ldots, N-1$ correspond to equally-spaced points on the upper half of the unit circle (when $N$ is even, they avoid the points $\pm 1$). Their symmetry about the imaginary axis ensures that horizontal projections cancel. This is precisely the discrete analog of the orthogonality integral $\int_0^\pi \cos(m\theta) \, d\theta = 0$ for $m > 0$.

\chapter{Power Sum Invariance}

We now prove that the power sums of the rotated projections $r_k(\varphi) = \cos(\varphi + 2\pi k/N)$ are independent of $\varphi$. This remarkable invariance is the heart of our main theorem: it shows that rotating the configuration does not change the polynomial (except for the constant term).

\section{Strategy via binomial expansion}

The key technique is to expand $\cos^j(\varphi + 2\pi k/N)$ using Euler's formula:
\begin{equation}
\cos\theta = \frac{\e^{i\theta} + \e^{-i\theta}}{2}.
\end{equation}
Raising to the $j$-th power and expanding binomially yields
\begin{equation}
\cos^j\theta = \frac{1}{2^j} \sum_{\ell=0}^j \binom{j}{\ell} \e^{i(j - 2\ell)\theta}.
\end{equation}
\lean{exp_add_exp_pow}
\leanok
Summing over $k = 0, \ldots, N-1$ with $\theta = \varphi + 2\pi k/N$, each term with $j - 2\ell \not\equiv 0 \pmod{N}$ contributes zero by discrete orthogonality (Lemma~\ref{lem:sum_cos_m_phi_vanishes}). Only the term with $j - 2\ell \equiv 0 \pmod{N}$ survives, and this term is independent of $\varphi$.

\section{Computational examples}

Before proving the general case, we illustrate the invariance with explicit calculations for small values.

\subsection{Example: $N = 3$, $j = 2$}

For $N = 3$ and $j = 2$, the power sum is
\begin{equation}
S_2(\varphi) = \sum_{k=0}^2 \cos^2\left(\varphi + \frac{2\pi k}{3}\right).
\end{equation}
Using $\cos^2\theta = (1 + \cos(2\theta))/2$:
\begin{equation}
S_2(\varphi) = \frac{3}{2} + \frac{1}{2} \sum_{k=0}^2 \cos\left(2\varphi + \frac{4\pi k}{3}\right).
\end{equation}
The sum vanishes by Lemma~\ref{lem:sum_cos_m_phi_vanishes} (with $m = 2 < N = 3$), giving $S_2(\varphi) = 3/2$ for all $\varphi$.

\subsection{Example: $N = 4$, $j = 3$}

For $N = 4$ and $j = 3$:
\begin{equation}
S_3(\varphi) = \sum_{k=0}^3 \cos^3\left(\varphi + \frac{\pi k}{2}\right).
\end{equation}
Using the identity $\cos^3\theta = \frac{3\cos\theta + \cos(3\theta)}{4}$:
\lean{cos_cube_formula}
\leanok
\begin{equation}
S_3(\varphi) = \frac{3}{4}\sum_{k=0}^3 \cos\left(\varphi + \frac{\pi k}{2}\right) + \frac{1}{4}\sum_{k=0}^3 \cos\left(3\varphi + \frac{3\pi k}{2}\right).
\end{equation}
Both sums vanish (first sum by Lemma~\ref{lem:sum_cos_phi_vanishes}, second sum by Lemma~\ref{lem:sum_cos_m_phi_vanishes} with $m = 3 < N = 4$), giving $S_3(\varphi) = 0$ for all $\varphi$.

\subsection{Example: $N = 5$, $j = 2$}

For $N = 5$ and $j = 2$:
\begin{equation}
S_2(\varphi) = \sum_{k=0}^4 \cos^2\left(\varphi + \frac{2\pi k}{5}\right) = \frac{5}{2} + \frac{1}{2}\sum_{k=0}^4 \cos\left(2\varphi + \frac{4\pi k}{5}\right) = \frac{5}{2}.
\end{equation}
The trigonometric sum vanishes since $m = 2 < N = 5$.

\section{General power sum invariance}

\begin{theorem}[Power sum invariance]\label{lem:power_sum_invariance}
\lean{powerSumCos_invariant, sum_cos_pow_theta_independent}
\leanok
For any integers $N \geq 1$, $j$ with $1 \leq j < N$, and any $\varphi \in \R$,
\begin{equation}
\sum_{k=0}^{N-1} \cos^j\left(\varphi + \frac{2\pi k}{N}\right)
\end{equation}
is independent of $\varphi$.
\lean{powerSumCos_invariant_j2, powerSumCos_invariant_j3, powerSumCos_invariant_j4, powerSumCos_invariant_j5, powerSumCos_invariant_j6}
\leanok
\lean{cos_cube_formula, cos_four_formula, cos_five_formula, cos_six_formula}
\leanok
\uses{lem:sum_cos_m_phi_vanishes, sum_cos_pow_eq_sum_binomial, exp_add_exp_pow}

\textit{(See \href{/docs/ChebyshevCircles/PowerSums.html\#powerSumCos_invariant}{API Documentation} | \href{../paper/chebyshev_circles.pdf}{Paper Theorem 4.1})}
\end{theorem}

\begin{proof}[Proof sketch]
Expand $\cos^j(\varphi + 2\pi k/N)$ using $\cos\theta = (\e^{i\theta} + \e^{-i\theta})/2$. The binomial expansion yields
\begin{equation}
\sum_{k=0}^{N-1} \cos^j\left(\varphi + \frac{2\pi k}{N}\right) = \frac{1}{2^j} \sum_{\ell=0}^j \binom{j}{\ell} \sum_{k=0}^{N-1} \e^{i(j-2\ell)(\varphi + 2\pi k/N)}.
\end{equation}
Factor out $\e^{i(j-2\ell)\varphi}$:
\begin{equation}
= \frac{1}{2^j} \sum_{\ell=0}^j \binom{j}{\ell} \e^{i(j-2\ell)\varphi} \sum_{k=0}^{N-1} \e^{2\pi i (j-2\ell)k/N}.
\end{equation}
By Lemma~\ref{lem:discrete_orthogonality}, the inner sum is $N$ if $N \mid (j - 2\ell)$, and $0$ otherwise. Since $1 \leq j < N$, we have $|j - 2\ell| \leq j < N$ for all $\ell \in \{0, \ldots, j\}$. The only way $N \mid (j - 2\ell)$ is if $j - 2\ell = 0$, i.e., $j = 2\ell$.

If $j$ is odd, no such $\ell$ exists, and the entire sum is zero (independent of $\varphi$). If $j$ is even, only the term $\ell = j/2$ contributes, yielding
\begin{equation}
\frac{N}{2^j} \binom{j}{j/2},
\end{equation}
which is independent of $\varphi$.
\lean{sum_cos_pow_eq_sum_binomial}
\leanok
\end{proof}

The formalization establishes this for small values of $j$ explicitly ($j = 2, 3, 4, 5, 6$) and then proves the general case using the binomial argument above.
\lean{sum_pow_primitive_root_mul, sum_primitive_root_pow_mul}
\leanok

\chapter{From Power Sums to Polynomial Coefficients}

Having established that the power sums are $\varphi$-independent, we now apply Newton's identities to deduce that the polynomial coefficients (except the constant term) are also $\varphi$-independent.

\section{Newton's identities: the algebraic bridge}

Recall that for roots $\alpha_1, \ldots, \alpha_N$, Newton's identities recursively determine the elementary symmetric polynomials $e_k$ from the power sums $p_j = \sum \alpha_i^j$:
\begin{equation}
k \cdot e_k = \sum_{i=1}^k (-1)^{i-1} e_{k-i} p_i.
\end{equation}
\lean{MvPolynomial.mul_esymm_eq_sum, multiset_newton_identity}
\leanok
Since this is a recursive formula starting from $e_0 = 1$, if the power sums $p_1, \ldots, p_N$ are independent of $\varphi$, then so are all elementary symmetric polynomials $e_1, \ldots, e_N$. By Vieta's formulas, these determine the polynomial coefficients (the coefficient of $x^{N-k}$ is $(-1)^k e_k$ in the monic polynomial).

\begin{lemma}\label{lem:esymm_invariant}
\lean{esymm_rotated_roots_invariant}
\leanok
Let $r_k(\varphi) = \cos(\varphi + 2\pi k/N)$ for $k = 0, \ldots, N-1$. Then for any $m$ with $1 \leq m \leq N$, the elementary symmetric polynomial
\begin{equation}
e_m(r_0(\varphi), \ldots, r_{N-1}(\varphi))
\end{equation}
is independent of $\varphi$.
\uses{lem:power_sum_invariance, thm:newton_identities}
\end{lemma}

\begin{proof}
By Theorem~\ref{lem:power_sum_invariance}, the power sums $p_j = \sum_{k=0}^{N-1} r_k(\varphi)^j$ are independent of $\varphi$ for $1 \leq j < N$. For $j = N$, the power sum may depend on $\varphi$, but Newton's identities for $e_m$ with $m \leq N$ only involve $p_1, \ldots, p_m \leq p_N$. Since $m \leq N$, we use $p_j$ for $j < N$, which are all invariant. Thus by induction on $m$, each $e_m$ is independent of $\varphi$.
\lean{esymm_eq_of_psum_eq}
\leanok
\end{proof}

\begin{theorem}\label{thm:coefficients_independent}
\lean{constant_term_only_varies}
\leanok
For any $N \geq 1$ and any $k$ with $1 \leq k \leq N$, the coefficient of $x^k$ in $S_N(x; \varphi)$ is independent of $\varphi$.
\uses{lem:power_sum_invariance, thm:newton_identities, esymm_rotated_roots_invariant}

\textit{(See Theorem 5.1 in \href{../paper/chebyshev_circles.pdf}{the paper})}
\end{theorem>

\begin{proof}
The scaled polynomial $S_N(x; \varphi) = 2^{N-1} P_N(x; \varphi)$, where $P_N$ is the monic polynomial with roots $r_k(\varphi)$. By Lemma~\ref{lem:esymm_invariant}, the elementary symmetric polynomials of the roots are $\varphi$-independent. By Vieta's formulas, the coefficients of $P_N$ (except possibly the constant term) are $\varphi$-independent. Scaling by $2^{N-1}$ preserves this property.
\lean{polynomial_coeff_eq_of_esymm_eq}
\leanok
\end{proof}

\section{Why the constant term varies}

The constant term of $P_N(x; \varphi)$ is $(-1)^N e_N = (-1)^N \prod_{k=0}^{N-1} r_k(\varphi)$. While the product of roots can vary with $\varphi$, Newton's identities only constrain $e_N$ using $p_1, \ldots, p_N$. Since $p_N$ (the sum of $N$-th powers) may depend on $\varphi$ when $j = N \geq N$, there is no contradiction. Indeed, explicit calculation shows the constant term does vary.
\lean{scaledPolynomial_constantTerm_varies}
\leanok

\chapter{Proof of Main Theorem}

We now assemble the pieces to prove Theorem~\ref{thm:main}.

\section{Chebyshev roots and their power sums}

To prove that $S_N(x; \varphi) = T_N(x) + c(\varphi)$, we must show that the roots $r_k(\varphi)$ and the Chebyshev roots $\xi_k$ have matching power sums. We already know the rotated roots have $\varphi$-independent power sums. Now we compute the power sums of the Chebyshev roots explicitly.

\begin{lemma}\label{lem:chebyshev_roots_characterization}
\lean{chebyshevRoot, chebyshevRootsList}
\leanok
The roots of the Chebyshev polynomial $T_N(x)$ are
\begin{equation}
\xi_k = \cos\left(\frac{(2k + 1)\pi}{2N}\right), \quad k = 0, 1, \ldots, N-1.
\end{equation}
These are $N$ distinct real numbers in the interval $(-1, 1)$.
\lean{chebyshevRoot_in_Icc, chebyshevRoots_distinct}
\leanok
\uses{prop:chebyshev_properties}
\end{lemma}

\begin{proof}
By the defining relation $T_N(\cos\theta) = \cos(N\theta)$, we have $T_N(\xi_k) = 0$ if and only if $N\theta_k = (2k + 1)\pi/2$ for some integer $k$, i.e., $\theta_k = (2k + 1)\pi/(2N)$. Since $\theta_k \in (0, \pi)$ for $k \in \{0, \ldots, N-1\}$, the values $\xi_k = \cos\theta_k$ are distinct and lie in $(-1, 1)$.
\lean{chebyshev_T_eval_chebyshevRoot, chebyshev_T_eval_eq_zero_iff}
\leanok
\end{proof}

\subsection{Computational examples: Chebyshev roots}

\begin{example}[$N = 3$]
The roots of $T_3(x) = 4x^3 - 3x$ are:
\begin{align*}
\xi_0 &= \cos(\pi/6) = \frac{\sqrt{3}}{2} \approx 0.866, \\
\xi_1 &= \cos(\pi/2) = 0, \\
\xi_2 &= \cos(5\pi/6) = -\frac{\sqrt{3}}{2} \approx -0.866.
\end{align*}
Power sums:
\begin{align*}
p_1 &= \xi_0 + \xi_1 + \xi_2 = 0, \\
p_2 &= \xi_0^2 + \xi_1^2 + \xi_2^2 = \frac{3}{4} + 0 + \frac{3}{4} = \frac{3}{2}.
\end{align*}
Note that $p_2 = \frac{3}{2} = \frac{N}{2^j}\binom{j}{j/2} = \frac{3}{4} \cdot 2 = \frac{3}{2}$ when $j = 2$.
\end{example}

\begin{example}[$N = 4$]
The roots of $T_4(x) = 8x^4 - 8x^2 + 1$ are:
\begin{align*}
\xi_0 &= \cos(\pi/8) \approx 0.924, \\
\xi_1 &= \cos(3\pi/8) \approx 0.383, \\
\xi_2 &= \cos(5\pi/8) \approx -0.383, \\
\xi_3 &= \cos(7\pi/8) \approx -0.924.
\end{align*}
Power sums:
\begin{align*}
p_1 &= 0 \quad \text{(sum of symmetric points)}, \\
p_2 &= 2\cos^2(\pi/8) + 2\cos^2(3\pi/8) = 2 \cdot \frac{2 + \sqrt{2}}{4} + 2 \cdot \frac{2 - \sqrt{2}}{4} = 2, \\
p_3 &= 0 \quad \text{(odd power, symmetric roots)}.
\end{align*}
\end{example}

\begin{example}[$N = 5$]
The roots of $T_5(x) = 16x^5 - 20x^3 + 5x$ are:
\begin{align*}
\xi_0 &= \cos(\pi/10) \approx 0.951, \\
\xi_1 &= \cos(3\pi/10) \approx 0.588, \\
\xi_2 &= \cos(\pi/2) = 0, \\
\xi_3 &= \cos(7\pi/10) \approx -0.588, \\
\xi_4 &= \cos(9\pi/10) \approx -0.951.
\end{align*}
Power sums: $p_1 = 0$, $p_2 = 5/2$, $p_3 = 0$, $p_4 = 15/8$.
\end{example}

\begin{lemma}\label{lem:chebyshev_roots}
\lean{chebyshev_roots_powersum_value}
\leanok
Let $\xi_k = \cos((2k + 1)\pi/(2N))$ for $k = 0, \ldots, N-1$ denote the roots of $T_N(x)$. Then for any $1 \leq j < N$,
\begin{equation}
\sum_{k=0}^{N-1} \xi_k^j = \begin{cases} \frac{N}{2^j} \binom{j}{j/2} & \text{if } j \text{ is even}, \\ 0 & \text{if } j \text{ is odd.} \end{cases}
\end{equation}
\lean{ChebyshevCircles.sum_cos_pow_chebyshev_binomial}
\leanok
\uses{lem:chebyshev_angle_sum_odd, lem:chebyshev_angle_sum_even, exp_add_exp_pow_chebyshev}

\textit{(See Lemma 6.1 in \href{../paper/chebyshev_circles.pdf}{the paper})}
\end{lemma}

\begin{proof}[Proof sketch]
The proof parallels that of Theorem~\ref{lem:power_sum_invariance}. Expand $\xi_k^j = \cos^j((2k+1)\pi/(2N))$ using the binomial formula:
\begin{equation}
\cos^j\theta = \frac{1}{2^j}\sum_{\ell=0}^j \binom{j}{\ell} \e^{i(j-2\ell)\theta}.
\end{equation}
\lean{cos_as_exp_chebyshev, exp_add_exp_pow_chebyshev}
\leanok
Summing over $k$ and using Lemmas~\ref{lem:chebyshev_angle_sum_odd} and~\ref{lem:chebyshev_angle_sum_even}, only the term with $j - 2\ell = 0$ survives (when $j$ is even). This gives the stated formula.
\end{proof}

\begin{theorem}[Power sum equality]\label{thm:power_sum_equality}
\lean{general_powersum_equality, rotated_roots_powersum_value}
\leanok
For $1 \leq j < N$,
\begin{equation}
\sum_{k=0}^{N-1} \cos^j\left(\varphi + \frac{2\pi k}{N}\right) = \sum_{k=0}^{N-1} \cos^j\left(\frac{(2k + 1)\pi}{2N}\right).
\end{equation}
\uses{lem:power_sum_invariance, lem:chebyshev_roots}

\textit{(See Theorem 6.2 in \href{../paper/chebyshev_circles.pdf}{the paper})}
\end{theorem}

\begin{proof}
Both sides are independent of $\varphi$ by Theorem~\ref{lem:power_sum_invariance}. Evaluating at any particular value of $\varphi$ establishes equality. The formalization uses $\varphi = \pi/(2N)$ as the bridge: the rotated roots at this angle coincide with a permutation of the Chebyshev roots, making the equality manifest.
\lean{realProjectionsList_powersum, multiset_powersum_realProjectionsList}
\leanok
\end{proof}

\section{Completion of the proof}

We now have all the ingredients to prove the main theorem.

\begin{proof}[Proof of Theorem~\ref{thm:main}]
Let $r_k(\varphi) = \cos(\varphi + 2\pi k/N)$ for $k = 0, \ldots, N-1$ be the roots of $P_N(x; \varphi)$, and let $\xi_k = \cos((2k + 1)\pi/(2N))$ be the roots of $T_N(x)$.
\lean{realProjectionsList, chebyshevRootsList}
\leanok

By Theorem~\ref{thm:power_sum_equality}, the power sums of $\{r_k(\varphi)\}$ and $\{\xi_k\}$ are equal for all $j$ with $1 \leq j < N$:
\begin{equation}
\sum_{i=0}^{N-1} r_i(\varphi)^j = \sum_{i=0}^{N-1} \xi_i^j.
\end{equation}

By Newton's identities (Theorem~\ref{thm:newton_identities}), the elementary symmetric polynomials $e_k$ for $k = 1, \ldots, N$ are determined recursively from the power sums $p_1, \ldots, p_k$. Since the power sums match for $j < N$, and the recursion for $e_k$ with $k \leq N$ only depends on $p_1, \ldots, p_k < p_N$, the elementary symmetric polynomials match:
\begin{equation}
e_k(r_0(\varphi), \ldots, r_{N-1}(\varphi)) = e_k(\xi_0, \ldots, \xi_{N-1}), \quad k = 1, \ldots, N.
\end{equation}
\lean{esymm_rotated_roots_invariant}
\leanok

The monic polynomial $P_N(x; \varphi)$ has roots $r_k(\varphi)$ and can be written as
\begin{equation}
P_N(x; \varphi) = \prod_{k=0}^{N-1}(x - r_k(\varphi)) = \sum_{k=0}^N (-1)^k e_k \, x^{N-k},
\end{equation}
where $e_k = e_k(r_0(\varphi), \ldots, r_{N-1}(\varphi))$ are the elementary symmetric polynomials.
\lean{polynomialFromRealRoots, unscaledPolynomial_monic}
\leanok

Similarly, the Chebyshev polynomial $T_N(x)$ has leading coefficient $2^{N-1}$ and roots $\xi_k$, so
\begin{equation}
T_N(x) = 2^{N-1} \prod_{k=0}^{N-1}(x - \xi_k).
\end{equation}
\lean{chebyshev_T_leadingCoeff}
\leanok

By Vieta's formulas, the coefficient of $x^{N-k}$ in $P_N(x; \varphi)$ is $(-1)^k e_k(r_0, \ldots, r_{N-1})$, and the coefficient of $x^{N-k}$ in $T_N(x)/2^{N-1}$ is $(-1)^k e_k(\xi_0, \ldots, \xi_{N-1})$.
\lean{Multiset.prod_X_sub_C_coeff}
\leanok

Since the elementary symmetric polynomials match for $k = 1, \ldots, N$, we have $P_N(x; \varphi)$ and $T_N(x)/2^{N-1}$ have the same coefficients for all degrees $k = 1, \ldots, N$. They may differ only in the constant term (degree 0).

Multiplying both by $2^{N-1}$, the scaled polynomial
\begin{equation}
S_N(x; \varphi) = 2^{N-1} P_N(x; \varphi)
\end{equation}
and $T_N(x)$ have the same coefficients for degrees $k = 1, \ldots, N$. They may differ in the constant term.
\lean{scaledPolynomial_leadingCoeff, scaledPolynomial_degree}
\leanok

Hence there exists $c(\varphi) \in \R$ such that
\begin{equation}
S_N(x; \varphi) = T_N(x) + c(\varphi),
\end{equation}
where $c(\varphi)$ is the difference in constant terms.
\lean{rotated_roots_yield_chebyshev}
\leanok

Moreover, since the coefficients of $x^k$ for $k \geq 1$ match exactly, we have coefficient-wise equality:
\begin{equation}
[x^k] S_N(x; \varphi) = [x^k] T_N(x) \quad \text{for all } k \geq 1.
\end{equation}
\lean{rotated_roots_coeffs_match_chebyshev, scaledPolynomial_matches_chebyshev_at_zero}
\leanok
This completes the proof of Theorem~\ref{thm:main}.
\end{proof}

\section{Explicit verification for small N}

To build intuition, we verify the theorem explicitly for small values of $N$.

\begin{example}[$N = 3$, $\varphi = 0$]
The rotated roots at $\varphi = 0$ are $r_k = \cos(2\pi k/3)$ for $k = 0, 1, 2$:
\begin{equation}
r_0 = 1, \quad r_1 = \cos(2\pi/3) = -\frac{1}{2}, \quad r_2 = \cos(4\pi/3) = -\frac{1}{2}.
\end{equation}
The unscaled polynomial is
\begin{equation}
P_3(x; 0) = (x - 1)\left(x + \frac{1}{2}\right)^2 = x^3 - \frac{3x}{2} + \frac{1}{4}.
\end{equation}
Scaling by $2^{N-1} = 4$ gives
\begin{equation}
S_3(x; 0) = 4x^3 - 6x + 1.
\end{equation}
This differs from $T_3(x) = 4x^3 - 3x$ by the constant $c(0) = 1$, as predicted.
\end{example}

\begin{example}[$N = 4$, $\varphi = 0$]
The rotated roots at $\varphi = 0$ are $r_k = \cos(\pi k/2)$ for $k = 0, 1, 2, 3$:
\begin{equation}
r_0 = 1, \quad r_1 = 0, \quad r_2 = -1, \quad r_3 = 0.
\end{equation}
The unscaled polynomial is
\begin{equation}
P_4(x; 0) = (x - 1)(x + 1) x^2 = x^4 - x^2.
\end{equation}
Scaling by $2^3 = 8$ gives
\begin{equation}
S_4(x; 0) = 8x^4 - 8x^2.
\end{equation}
This differs from $T_4(x) = 8x^4 - 8x^2 + 1$ by the constant $c(0) = 1$.
\end{example}
